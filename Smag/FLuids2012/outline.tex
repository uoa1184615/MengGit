\documentclass[12pt,a5paper]{article}
\usepackage[margin=6mm,bottom=15mm]{geometry}
\usepackage{url}
\usepackage{graphicx}
\usepackage{subfigure}
\usepackage{amsmath,defns,reducecode}
\usepackage{natbib}
\def\harvardurl#{}
\usepackage{hyperref} 
\hypersetup{pdftex,
            backref=true,
            hyperindex=true,
            colorlinks=true,
            citecolor=blue,
            bookmarks=true,
            breaklinks=true}

\title{Outline of Modelling 3D turbulent floods based on the Smagorinski large eddy closure}
\author{Meng Cao}
%\date{}

\begin{document}
\maketitle            
Bellow is the outline for the work of Modelling 3D turbulent floods based on the Smagorinski large eddy closure.           

\section{Introduction}

Describe conventional depth-averaged turbulence models briefly or introduce the low order governing equations of the proposed turbulent modelling?
                        
\section{The description of the turbulent modelling}
\begin{itemize}
\item Governing partial differential equations
\item Smagorinski large eddy closure
\item Boundary conditions
\end{itemize}
           
\section{Center manifold theory supports the modelling}            
 
Is it necessary to mention the Center manifold theory? Since it was described in the work of~\cite{Roberts2008} and~\cite{Georgiev2008}.            
            
            
\section{Modelling turbulent flows in straight and meandering open channels}            

Use the model simulate turbulence flows in straight and meandering open channels. Compare the results with relevant published work~\cite[e.g.]{Bousmar2002,Liu2009}.  
 
\section{Conclusion} 
 
\bibliographystyle{agsm}
\bibliography{Turbulence}
\end{document}