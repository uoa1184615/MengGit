\documentclass[12pt,a5paper]{article}
\usepackage[margin=6mm,bottom=15mm]{geometry}
\usepackage{url}
\usepackage{graphicx}
\usepackage{subfigure}
\usepackage{amsmath}
\usepackage{natbib}
\def\harvardurl#{}
\usepackage{hyperref} 
\hypersetup{pdftex,
            backref=true,
            hyperindex=true,
            colorlinks=true,
            citecolor=blue,
            bookmarks=true,
            breaklinks=true}
%\pagestyle{headings}

\newcommand{\D}{\partial}
\newcommand{\B}{\textbf}

\title{Model 3D turbulent floods based upon the Smagorinski large eddy closure}
%\author{Meng Cao}
\date{}

\begin{document}
\maketitle

\section*{Abstract}


Rivers, floods and tsunamis are often very turbulent. Conventional models of such environmental fluids are typically based on depth-averaged inviscid irrotational flow equations. We explore changing such a base to the turbulent Smagorinski large eddy closure. The aim is to more appropriately model the fluid dynamics of such complex environmental fluids by using such a turbulent closure. Large changes in fluid depth are allowed. Computer algebra constructs the slow manifold of the flow in terms of the fluid depth~$h$ and the mean turbulent lateral velocities~$\bar u$ and~$\bar v$. The major challenge is to deal with the nonlinear stress tensor in the Smagorinski closure. The model integrates the effects of inertia, self-advection, bed drag, gravitational forcing and turbulent dissipation with minimal assumptions. Although the resultant model is close to established models, the real outcome is creating a sound basis for the modelling so others, in their modelling of more complex situations, can systematically include more complex physical processes.

\paragraph{Keywords} turbulent flood, tsunami, Smagorinski closure, channel flows














\bibliographystyle{agsm}
\bibliography{Turbulence}
\end{document}