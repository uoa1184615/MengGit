\documentclass[12pt,a5paper]{article}
\usepackage[margin=6mm,bottom=15mm]{geometry}
\usepackage{url}
\usepackage{graphicx}
\usepackage{subfigure}
\usepackage{amsmath}
\usepackage{natbib}
\def\harvardurl#{}
\usepackage{hyperref} 
\hypersetup{pdftex,
            backref=true,
            hyperindex=true,
            colorlinks=true,
            citecolor=blue,
            bookmarks=true,
            breaklinks=true}
%\pagestyle{headings}

\newcommand{\D}{\partial}
\newcommand{\B}{\textbf}

\title{Multiscale modelling couples patches of wave-like simulations}
\author{Meng Cao}
\date{\today}

\begin{document}
\maketitle

\section*{Abstract}

A multiscale model is proposed to reduce the expensive numerical simulations of complicated waves over large spatial domains.
The multiscale model is built from given microscale simulations of complicated physical processes such as sea ice or turbulent shallow water. 
Our long term aim is to enable macroscale simulations obtained by coupling small patches of simulations together over large physical distances. 
This initial work explores the coupling of patch simulations of wave-like~\textsc{pde}s.
With the line of development being to water waves we discuss the dynamics of two complementary fields called the `depth'~$h$ and `velocity'~$u$.
A staggered grid is used for the microscale simulation of the depth~$h$ and velocity~$u$. 
We introduce a macroscale staggered grid to couple the microscale patches. 
Linear or quadratic interpolation provides boundary conditions on the field in each patch.
Linear analysis of the whole coupled multiscale system establishes that the resultant macroscale dynamics is appropriate.
Numerical simulations support the linear analysis.
This multiscale method should empower the feasible computation of large scale simulations of wave-like dynamics with complicated underlying physics.




















\bibliographystyle{agsm}
\bibliography{Turbulence}
\end{document}