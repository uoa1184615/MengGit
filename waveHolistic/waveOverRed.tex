\IfFileExists{ajr.sty}
{\documentclass[11pt,a5paper]{article}}
{\documentclass[12pt,a4paper]{refart}}

\usepackage{amsmath,defns,reducecode}
\IfFileExists{ajr.sty}{\usepackage{ajr}}{}

\newcommand{\cG}{\ensuremath{\mathcal G}}
\newcommand{\shift}{\text{\small$\mathcal E$}}
\newcommand{\sech}{\operatorname{sech}}
\newcommand{\jp}{_{j+1}}
\newcommand{\jm}{_{j-1}}
\newcommand{\jpm}{_{j\pm1}}
\newcommand{\jmp}{_{j\mp1}}
\newcommand{\jpmh}{_{j\pm1/2}}
\newcommand{\jmm}{_{j-2}}
\newcommand{\jpp}{_{j+2}}

\title{Holistic discretisation of wave-like PDEs, II}
\author{Tony Roberts \and Meng Cao}
\date{\today}

\begin{document}

\maketitle


\section{Introduction}

Try to develop good numerics of wave-like \pde{}s using
a staggered element approach. For `small' parameter~$\nu$, the \pde{}s for fields $h(x,t)$~and~$u(x,t)$ are among
\begin{align*}&
\D th=-\D xu\,,
\\&
\D tu=-\D xh-\nu u+\nu\DD xu-\nu u\D xu\,.
\end{align*}
To be solved on elements centred on~$X_j$, $X_j=jD$ say, with some coupling condition.  The difference here is that we let the elements overlap so the $j$th~element is the interval $E_j=(X\jm,X\jp)$.

Because of even/odd symmetry I think it is more convenient to imagine two fields, each on overlapping elements, for each physical field: in element~$E_j$ introduce $h_j(x,t)$, $h'_j(x,t)$, $u_j(x,t)$ and~$u'_j(x,t)$.  I aim to eventuates that even-undashed fields interact with odd-undashed fields, and vice versa, but that the two sets of fields do not interact with the other.  The \pde{}s are then
\begin{align*}&
j\text{ odd (even)}&&j\text{ even (odd)}
\\&
\D t{h'_j}=-\D x{u_j}\,,
&&\D t{h_j}=-\D x{u'_j}\,,
\\&
\D t{u_j}=-\D x{h'_j}-\nu u_j\,,
&&\D t{u'_j}=-\D x{h_j}-\nu u'_j\,.
\end{align*}


Here I propose the coupling condition on the fields, $j$~even (odd), of
\begin{align*}&
(1-\rat12\gamma)\big[h_j(X\jp,t)-h_j(X\jm)\big]
=\rat12\gamma\big[h\jpp(X\jp,t)-h\jmm(X\jm,t)\big], 
\\&
u'_j(X_j,t)=\rat12\big[u\jp(X_j,t)+u\jm(X_j,t)\big],
\end{align*}
and correspondingly couple the fields, $j$~odd (even), with
\begin{align*}&
(1-\rat12\gamma)\big[u_j(X\jp,t)-u_j(X\jm)\big]
=\rat12\gamma\big[u\jpp(X\jp,t)-u\jmm(X\jm,t)\big], 
\\&
h'_j(X_j,t)=\rat12\big[h\jp(X_j,t)+h\jm(X_j,t)\big],
\end{align*}
Lastly, define the amplitudes to be 
\begin{equation*}
H_j=h_j(X_j) \qtq{and} U_j=u_j(X_j),
\end{equation*}
respectively for $j$~even and odd (odd and even).
Be careful with the dashes.



\section{Eigenvalue analysis}

Assume~$\nu=0$\,, thus there is no dissipation (bed drag) in the \textsc{pde}s. Seek solutions in exponential form
\begin{eqnarray}
&&u_j(x,t)=u(\xi)e^{\lambda t+ikj}\,,\label{sol:u}\\
&&h_j(x,t)=h(\xi)e^{\lambda t+ikj}\,,\label{sol:h}\\
&&u'_j(x,t)=u'(\xi)e^{\lambda t+ikj}\,,\label{sol:u'}\\
&&h'_j(x,t)=h'(\xi)e^{\lambda t+ikj}\,,\label{sol:h'}
\end{eqnarray}
where $\xi=(x-X_j)/D$ so that~$\partial_x=\frac1D\partial_\xi$\,, and~$k$ is the lateral wavenumber.
In essence, $u(\xi),h(\xi),u'(\xi),h'(\xi)$ are Fourier transforms over the element index~$j$ of the corresponding fields.
Substituting these exponential forms into the \textsc{pde}s gives
\marginpar{Use my command \texttt{$\backslash$DD} to do second order partial derivatives, and \texttt{$\backslash$D} for first order partial derivative.}
\begin{equation}
\lambda^2u(\xi)=\frac{1}{D^2}\frac{\partial^2u(\xi)}{\partial\xi^2}
\quad\text{and}\quad
\lambda^2u'(\xi)=\frac{1}{D^2}\frac{\partial^2u'(\xi)}{\partial\xi^2}\,.
\label{eq:FTpde}
\end{equation}
Being constant coefficient we try solutions for the subgrid structure in terms of trigonometric functions:\marginpar{Use one line if reasonable.} 
\begin{equation}
u(\xi)=A\cos\ell\xi+B\sin\ell\xi \label{sol:uxi}
\quad\text{and}\quad
u'(\xi)=A'\cos\ell\xi+B'\sin\ell\xi\,,
\end{equation}
where~$\ell$ is the wavenumber of the subgrid structures.
Substitute the above~$u(\xi)$ and~$u'(\xi)$ into the \textsc{pde}s~\eqref{eq:FTpde}\marginpar{Always be more explicit: cross-reference precisely, not vaguely.} and obtain that the solutions $h(\xi)$~and~$h'(\xi)$ must take the forms 
\begin{equation}
h'(\xi)=\frac{\ell}{D\lambda}(A\sin\ell\xi-B\cos\ell\xi)
\quad\text{and}\quad
h(\xi)=\frac{\ell}{D\lambda}(A'\sin\ell\xi-B'\cos\ell\xi)\,.\label{sol:hxi}
\end{equation}

The coupling conditions~\eqref{??} proposed in the introduction indicate
\begin{align}&
(1-\frac12\gamma)\big[h(1)-h(-1)\big]
=\frac12\gamma\big[h(-1)e^{2ik}-h(1)e^{-2ik}\big], \label{bc:h}
\\&
u'(0)=\frac12\big[u(1)+u(-1)\big],\label{bc:u'}
\\&
(1-\frac12\gamma)\big[u(1)-u(-1)\big]
=\frac12\gamma\big[u(-1)e^{2ik}-u(1)e^{-2ik}\big], \label{bc:u}
\\&
h'(0)=\frac12\big[h(1)+h(-1)\big],\label{bc:h'}
\end{align}
Substitute the solutions forms~\eqref{??} of~$h(\xi)$,~$u(\xi)$,~$h'(\xi)$ and~$u'(\xi)$ into the above coupling conditions, rearrange and obtain\marginpar{Give a couple of words to describe what is coming, namely, here two equations} the two equations\marginpar{You have missed out some steps here as $B$ and $A'$ have disappeared.  Describe.}
\begin{eqnarray*}
&&\big[(4-2\gamma)+\gamma(e^{2ik}+e^{-2ik})\big]\cos\ell\sin\ell\, A+\gamma\cos\ell(e^{2ik}-e^{-2ik})\,B'=0\,,\\
&&-\gamma\cos\ell(e^{2ik}-e^{-2ik})\,A+\big[(4-2\gamma)+\gamma(e^{2ik}+e^{-2ik})\big]\cos\ell\sin\ell \,B'=0\,,
\end{eqnarray*}
Nontrivial solutions of these equations exist only when the coefficient matrix is singular.
Setting the determinant of the coefficient matrix equaling to zero gives the characteristic equation\marginpar{Use more words.}
\begin{equation}
\left\{\big[(4-2\gamma)+\gamma(e^{2ik}+e^{-2ik})\big]\cos\ell\sin\ell\right\}^2=\gamma^2\cos^2\ell(e^{2ik}-e^{-2ik})^2\,.
\end{equation}
Invoking \marginpar{Avoid "$\backslash[\cdots\backslash]$, use either \texttt{displaymath} or \texttt{equation*} environments.}
\[
e^{2ik}+e^{-2ik}=2\cos2k\quad \text{and}\quad(e^{2ik}-e^{-2ik})^2=-4\sin^22k\,,
\]
the characteristic equation becomes
\begin{equation}
\big[(2-\gamma)\sin\ell+\gamma\sin\ell\cos2k\pm\gamma\sin2k\big]\cos\ell=0\,.
\label{eq:chareqn}
\end{equation}

Two possibilities arise. 
\begin{enumerate}
\item If~$\cos\ell=0$\,, then~$\ell=\pi/2,3\pi/2,5\pi/2,\ldots$ for all coupling~$\gamma$.  ??\marginpar{Interpret more; justify a little why you say ``good''.} which demonstrate the subgrid structures are good.
\item If~$(2-\gamma)\sin\ell+\gamma\sin\ell\cos2k\pm\gamma\sin2k=0$\,, 
in the decoupled case, $\gamma=0$\,, obtain~$\sin\ell=0$, which indicates~$\ell=0,\pi,2\pi,\ldots$. \marginpar{Interpret a little more.}
In the case of full coupling, $\gamma=1$, obtain \marginpar{Avoid "$\backslash[\cdots\backslash]$.}
  \[
     \sin\ell+\sin\ell\cos2k\pm\sin2k=0\,.        
  \]
Triangular transforms of above equation gives\marginpar{Following is wrong because $\sin\ell$ factors: presume the second one is a $\sin k$??}
\begin{equation}
(\sin\ell\cos k\pm\sin\ell)\cos k=0\,,
\end{equation}
 which indicates a special case of $\cos k=0$ such that~$k=\pi/2,3\pi/2,\ldots$, or $\sin\ell\cos k\pm\sin\ell=0$ such that 
\begin{equation}
\sin\ell=\pm\tan k\,.
\end{equation}
\end{enumerate}

\marginpar{Now, as well as discussing the above further, the other thing to do here is to check that the expansion in small coupling~$\gamma$ of the characteristic equation~\eqref{eq:chareqn} agrees with my computer algebra of the next section??  May be best to update Reduce so that the command \texttt{in\_tex} works.}


\section{Computer algebra constructs the slow manifold}

\paragraph{Improve printing}
\begin{reduce}
on div; off allfac; on revpri;
linelength(64)$ factor dd,df; 
\end{reduce}

\paragraph{Avoid slow integration with specific operator}
Introduce the sign function to handle the derivative discontinuities across the centre of each element.  Define the integral operator to handle polynomials with sign functions, both indefinite ($\int_0^\xi d\xi$) and definite to $\xi=\verb|q|=\pm 1$ ($\int_0^qd\xi$).
\begin{reduce}
operator intx; linear intx;
let { intx(xi^~~p,xi)=>xi^(p+1)/(p+1)
    , intx(1,xi)=>xi
    , intx(xi^~~p,xi,~q)=>q^(p+1)/(p+1)
    , intx(1,xi,~q)=>q
    };
\end{reduce}


\paragraph{Introduce subgrid variable}
Introduced above is the subgrid variable $\xi=(x-X_j)/D$, $|\xi|<1$\,, in which the fields are described.
\begin{reduce}
depend xi,x;  let df(xi,x)=>1/dd;
\end{reduce}

\paragraph{Define evolving amplitudes}
Amplitudes are as $U_j(t)=u'_j(X_j,t)$ and $H_j(t)=h'_j(X_j,t)$, 
The difference here is that we take, say, even~$j$ to be the $u$-elements and odd~$j$ to be the $h$-elements.  Actually it should not matter which way around, or even if you regard the modelling as being of two disjoint systems (one one way and one the other).  The amplitudes depend upon time according to some approximation stored in \verb|gh|~and~\verb|gu|.
\begin{reduce}
operator hh; operator uu;
depend hh,t; depend uu,t;
let { df(hh(~k),t)=>sub(j=k,gh)
    , df(uu(~k),t)=>sub(j=k,gu)
    };
\end{reduce}


\paragraph{But solvability condition is coupled}
Now the evolution equations are coupled together.  By some symmetry we decouple the equations using this operator~\verb|ginv|.  However, I expect that some problems will not decouple (look for non-cancelling pollution by~\verb|ginv| operators).  In which case we have to accept that the DEs for the amplitudes are \emph{implicit} DEs using the following operator.  Let's define $\cG=E+E^{-1}$ so that $\cG F_j=F_{j+1}+F_{j-1}$\,.  Take $\cG^{-1}$ of this equation to deduce $\cG^{-1} F_{j\pm1}=F_j-\cG^{-1} F_{j\mp1}$\,, and change subscripts, $j\mapsto k\mp1$\,, to deduce $\cG^{-1} F_k=F_{k\mp1}-\cG^{-1} F_{k\mp2}$\,.  That is, we change an inverse of~\cG\ to one with subscript closer to $k=j$\,, or otherwise if we desire. Have here coded some quadratic transformations so we can resolve quadratic terms in the model, but I guess we also might want cubic.

The following causes a warning that \verb|~a| and \verb|~b| are declared operator, which is fine, but I cannot predefine them as operators so cannot avoid the warning.
\begin{reduce}
operator ginv; linear ginv;
let { df(ginv(~a,t),t)=>ginv(df(a,t),t)
    , ginv(~a(j+~k),t)=>a(j+k-1)-ginv(a(j+k-2),t) when k>1
    , ginv(~a(j+~k),t)=>a(j+k+1)-ginv(a(j+k+2),t) when k<0
    , ginv(~a(j+~k)^2,t)=>a(j+k-1)^2-ginv(a(j+k-2)^2,t) when k>1
    , ginv(~a(j+~k)^2,t)=>a(j+k+1)^2-ginv(a(j+k+2)^2,t) when k<0
    , ginv(~a(j+~~k)*~b(j+~~l),t)=> a(j+k-1)*b(j+l-1)
      -ginv(a(j+k-2)*b(j+l-2),t) when k+l>2
    , ginv(~a(j+~~k)*~b(j+~~l),t)=> a(j+k+1)*b(j+l+1)
      -ginv(a(j+k+2)*b(j+l+2),t) when k+l<-1
    };
\end{reduce}

\paragraph{Start with linear approximation}
The linear approximation is the usual piecewise constant fields in each element.  Except that the dashed fields are (surprisingly sensible) averages of the surrounding elements.
\begin{reduce}
hj:=hh(j); hdj:=(hh(j+1)+hh(j-1))/2;
uj:=uu(j); udj:=(uu(j+1)+uu(j-1))/2;
gh:=gu:=0;
\end{reduce}

Truncate the asymptotic series in coupling~$\gamma$ and any other parameter, such as~$\nu$.   The basic slow manifold model evolution only appears at odd powers of~$\gamma$, so choosing errors to be even power of~$\gamma$ is good.
\begin{reduce}
let gam^6=>0; factor gam;
gamma:=gam;
let nu^2=>0; factor nu; 
\end{reduce}


\paragraph{Iterate to a slow manifold}
Iterate to seek a solution, terminating only when residuals are zero to specified order.
\begin{reduce}
for it:=1:9 do begin
write "ITERATION = ",it;
\end{reduce}

Choose this order of updating fields from residuals due to the pattern of communication.

\paragraph{First} do the equations for the evolution of the dashed fields. {$j$ even}
\begin{reduce}
resud:=df(udj,t)+df(hj,x)+nu*udj-nu*df(udj,x,2);
write lengthresud:=length(resud);
reshb:=(1-gamma/2)*(sub(xi=+1,hj)-sub(xi=-1,hj))
    -gamma/2*(sub({j=j+2,xi=-1},hj)-sub({j=j-2,xi=+1},hj));
write lengthreshb:=length(reshb);
write
gu:=gu+(gud:=ginv(reshb/dd
    -intx(resud,xi,1)+intx(resud,xi,-1),t));
hj:=hj-dd*intx(resud+sub(j=j-1,gud)/2+sub(j=j+1,gud)/2,xi);
\end{reduce}

{$j$ odd}
\begin{reduce}
reshd:=df(hdj,t)+df(uj,x);
write lengthreshd:=length(reshd);
resub:=(1-gamma/2)*(sub(xi=+1,uj)-sub(xi=-1,uj))
    -gamma/2*(sub({j=j+2,xi=-1},uj)-sub({j=j-2,xi=+1},uj));
write lengthresub:=length(resub);
write
gh:=gh+(ghd:=ginv(resub/dd
    -intx(reshd,xi,1)+intx(reshd,xi,-1),t));
uj:=uj-dd*intx(reshd+sub(j=j-1,ghd)/2+sub(j=j+1,ghd)/2,xi);
\end{reduce}

\paragraph{Second} do the equations for the evolution of the undashed fields, to get spatial structure of dashed fields.
{$j$ even}
\begin{reduce}
resh:=df(hj,t)+df(udj,x);
write lengthresh:=length(resh);
resua:=-sub(xi=0,udj)
    +sub({j=j+1,xi=-1},uj)/2+sub({j=j-1,xi=+1},uj)/2;
write lengthresua:=length(resua);
udj:=udj+resua-dd*int(resh,xi);
\end{reduce}

{$j$ odd}
\begin{reduce}
resu:=df(uj,t)+df(hdj,x)+nu*uj-nu*df(uj,x,2);
write lengthresu:=length(resu);
resha:=-sub(xi=0,hdj)
    +sub({j=j+1,xi=-1},hj)/2+sub({j=j-1,xi=+1},hj)/2;
write lengthresha:=length(resha);
hdj:=hdj+resha-dd*intx(resu,xi);
\end{reduce}

\paragraph{Terminate the loop}
Exit the loop if all residuals are zero.
\begin{reduce}
  if {resh,reshd,resha,reshb,resu,resud,resua,resub}
  ={0,0,0,0,0,0,0,0} then write it:=it+100000;
  showtime;
end;    
\end{reduce}



\paragraph{Equivalent PDEs}
Finish by finding the equivalent \pde\ for the discretisation.  Since $\cG=E+E^{-1}=e^{D\partial}+e^{-D\partial}=2\cosh(D\partial)$ so $\cG^{-1}=\rat12\sech(D\partial)$.  Find the discretisation is consistent to an order in grid spacing~$D$ that increases with order of coupling~$\gamma$.
\begin{reduce}
let dd^8=>0;
depend uu,x; depend hh,x;
rules:={uu(j)=>uu, uu(j+~p)=>uu+(for n:=1:8 sum 
               df(uu,x,n)*(dd*p)^n/factorial(n))
       ,hh(j)=>hh, hh(j+~p)=>hh+(for n:=1:8 sum 
               df(hh,x,n)*(dd*p)^n/factorial(n)) 
       ,ginv(~a,t)=>1/2*(a-1/2*dd^2*df(a,x,2) 
       +5/24*dd^4*df(a,x,4) -61/720*dd^6*df(a,x,6)
       +277/8064*dd^8*df(a,x,8) )
       }$
ghde:=(gh where rules);
gude:=(gu where rules);
\end{reduce}



\paragraph{Draw graph of subgrid field}
The first plot call is a dummy that appears needed on my system for some unknown reason.
\begin{reduce}
plot(sin(xi),terminal=aqua);
u0:=sub(j=0,uj)$ u1:=sub(j=1,udj)$
u0:=(u0 where {nu=>0,gam=>1,uu(0)=>1,uu(~k)=>0 when k neq 0});
u1:=(u1 where {nu=>0,gam=>1,uu(0)=>1,uu(~k)=>0 when k neq 0});
plot({u0,u1},xi=(-4 .. 4),terminal=aqua);
\end{reduce}

\paragraph{Finish}\begin{reduce}end;\end{reduce}

\section{Sample output}
\begin{verbatim}
1: in_tex "waveOverRed.tex"$

*** ~a declared operator 

*** ~b declared operator 

hj := hh(j)

        1               1
hdj := ---*hh(1 + j) + ---*hh( - 1 + j)
        2               2

uj := uu(j)

        1               1
udj := ---*uu(1 + j) + ---*uu( - 1 + j)
        2               2

gh := gu := 0

gamma := gam

ITERATION = 1

lengthresud := 3

lengthreshb := 3

        -1          1               1
gu := dd  *gam*( - ---*hh(1 + j) + ---*hh( - 1 + j)) - nu*uu(j)
                    2               2

lengthreshd := 1

lengthresub := 3

        -1          1               1
gh := dd  *gam*( - ---*uu(1 + j) + ---*uu( - 1 + j))
                    2               2

lengthresh := 7

lengthresua := 5

lengthresu := 7

lengthresha := 5

Time: 20 ms

ITERATION = 2

lengthresud := 12

lengthreshb := 5

        -1          1               1                    -1    3
gu := dd  *gam*( - ---*hh(1 + j) + ---*hh( - 1 + j)) + dd  *gam
                    2               2

            1                1                1
      *( - ----*hh(1 + j) + ----*hh(3 + j) + ----*hh( - 1 + j)
            16               48               16

            1
         - ----*hh( - 3 + j)) - nu*uu(j)
            48

lengthreshd := 12

lengthresub := 5

        -1          1               1                    -1    3
gh := dd  *gam*( - ---*uu(1 + j) + ---*uu( - 1 + j)) + dd  *gam
                    2               2

            1                1                1
      *( - ----*uu(1 + j) + ----*uu(3 + j) + ----*uu( - 1 + j)
            16               48               16

            1
         - ----*uu( - 3 + j))
            48

lengthresh := 15

lengthresua := 5

lengthresu := 15

lengthresha := 5

Time: 10 ms

ITERATION = 3

lengthresud := 7

lengthreshb := 7

        -1          1               1                    -1    3
gu := dd  *gam*( - ---*hh(1 + j) + ---*hh( - 1 + j)) + dd  *gam
                    2               2

            1                1                1
      *( - ----*hh(1 + j) + ----*hh(3 + j) + ----*hh( - 1 + j)
            16               48               16

            1
         - ----*hh( - 3 + j)) - nu*uu(j)
            48

lengthreshd := 7

lengthresub := 7

        -1          1               1                    -1    3
gh := dd  *gam*( - ---*uu(1 + j) + ---*uu( - 1 + j)) + dd  *gam
                    2               2

            1                1                1
      *( - ----*uu(1 + j) + ----*uu(3 + j) + ----*uu( - 1 + j)
            16               48               16

            1
         - ----*uu( - 3 + j))
            48

lengthresh := 1

lengthresua := 9

lengthresu := 1

lengthresha := 9

Time: 10 ms

ITERATION = 4

lengthresud := 1

lengthreshb := 1

        -1          1               1                    -1    3
gu := dd  *gam*( - ---*hh(1 + j) + ---*hh( - 1 + j)) + dd  *gam
                    2               2

            1                1                1
      *( - ----*hh(1 + j) + ----*hh(3 + j) + ----*hh( - 1 + j)
            16               48               16

            1
         - ----*hh( - 3 + j)) - nu*uu(j)
            48

lengthreshd := 1

lengthresub := 1

        -1          1               1                    -1    3
gh := dd  *gam*( - ---*uu(1 + j) + ---*uu( - 1 + j)) + dd  *gam
                    2               2

            1                1                1
      *( - ----*uu(1 + j) + ----*uu(3 + j) + ----*uu( - 1 + j)
            16               48               16

            1
         - ----*uu( - 3 + j))
            48

lengthresh := 1

lengthresua := 1

lengthresu := 1

lengthresha := 1

it := 100004

Time: 10 ms

                           1               2
ghde :=  - df(uu,x)*gam - ---*df(uu,x,3)*dd *gam
                           6

            1               2    3     1                4
         + ---*df(uu,x,3)*dd *gam  - -----*df(uu,x,5)*dd *gam
            6                         120

            1                4    3     1                 6
         + ----*df(uu,x,5)*dd *gam  - ------*df(uu,x,7)*dd *gam
            12                         5040

            13                6    3
         + -----*df(uu,x,7)*dd *gam
            720

                                   1               2
gude :=  - nu*uu - df(hh,x)*gam - ---*df(hh,x,3)*dd *gam
                                   6

            1               2    3     1                4
         + ---*df(hh,x,3)*dd *gam  - -----*df(hh,x,5)*dd *gam
            6                         120

            1                4    3     1                 6
         + ----*df(hh,x,5)*dd *gam  - ------*df(hh,x,7)*dd *gam
            12                         5040

            13                6    3
         + -----*df(hh,x,7)*dd *gam
            720
\end{verbatim}
\end{document}
